\documentclass{article}
\usepackage[a4paper,margin=2cm]{geometry}
\usepackage[utf8]{inputenc}
\usepackage{timeline}
\usepackage{url}
\usepackage[UKenglish]{babel}
\usepackage{listings}
\lstset{
mathescape,
basicstyle=\scriptsize\ttfamily,
commentstyle=\ttfamily\color{gray},
numbers=left,
numberstyle=\ttfamily\color{gray}\footnotesize,
stepnumber=1,
numbersep=5pt,
backgroundcolor=\color{white},
showspaces=false,
showstringspaces=false,
showtabs=false,
frame=single,
tabsize=2,
captionpos=b,
breaklines=true,
breakatwhitespace=false,
title=\lstname,
escapeinside={},
keywordstyle={},
morekeywords={},
mathescape,
literate=
  {á}{{\'a}}1 {é}{{\'e}}1 {í}{{\'i}}1 {ó}{{\'o}}1 {ú}{{\'u}}1
  {Á}{{\'A}}1 {É}{{\'E}}1 {Í}{{\'I}}1 {Ó}{{\'O}}1 {Ú}{{\'U}}1
  {à}{{\`a}}1 {è}{{\`e}}1 {ì}{{\`i}}1 {ò}{{\`o}}1 {ù}{{\`u}}1
  {À}{{\`A}}1 {È}{{\'E}}1 {Ì}{{\`I}}1 {Ò}{{\`O}}1 {Ù}{{\`U}}1
  {ä}{{\"a}}1 {ë}{{\"e}}1 {ï}{{\"i}}1 {ö}{{\"o}}1 {ü}{{\"u}}1
  {Ä}{{\"A}}1 {Ë}{{\"E}}1 {Ï}{{\"I}}1 {Ö}{{\"O}}1 {Ü}{{\"U}}1
  {â}{{\^a}}1 {ê}{{\^e}}1 {î}{{\^i}}1 {ô}{{\^o}}1 {û}{{\^u}}1
  {Â}{{\^A}}1 {Ê}{{\^E}}1 {Î}{{\^I}}1 {Ô}{{\^O}}1 {Û}{{\^U}}1
  {œ}{{\oe}}1 {Œ}{{\OE}}1 {æ}{{\ae}}1 {Æ}{{\AE}}1 {ß}{{\ss}}1
  {ű}{{\H{u}}}1 {Ű}{{\H{U}}}1 {ő}{{\H{o}}}1 {Ő}{{\H{O}}}1
  {ç}{{\c c}}1 {Ç}{{\c C}}1 {ø}{{\o}}1 {å}{{\r a}}1 {Å}{{\r A}}1
  {€}{{\euro}}1 {£}{{\pounds}}1 {«}{{\guillemotleft}}1
  {»}{{\guillemotright}}1 {ñ}{{\~n}}1 {Ñ}{{\~N}}1 {¿}{{?`}}1
}

\author{Arne Meier\\\url{meier@thi.uni-hannover.de}}
\title{\texttt{timeline} Package}

\newcommand{\parameter}[1]{\langle\normalfont\textit{#1}\rangle}

\begin{document}
\maketitle
\begin{abstract}
	The \texttt{timeline} package provides an easy interface to create and maintain timelines.
	Its macros utilise the \texttt{tikzenvironment} and commands from the  \texttt{pgf} package.
	The date granularity for entries on a timebar is year and month.
\end{abstract}

\section{License}
This package is under the MIT License. Copyright (c) 2018 Arne Meier.

Permission is hereby granted, free of charge, to any person obtaining a copy
of this software and associated documentation files (the "Software"), to deal
in the Software without restriction, including without limitation the rights
to use, copy, modify, merge, publish, distribute, sublicense, and/or sell
copies of the Software, and to permit persons to whom the Software is
furnished to do so, subject to the following conditions:

The above copyright notice and this permission notice shall be included in all
copies or substantial portions of the Software.

THE SOFTWARE IS PROVIDED "AS IS", WITHOUT WARRANTY OF ANY KIND, EXPRESS OR
IMPLIED, INCLUDING BUT NOT LIMITED TO THE WARRANTIES OF MERCHANTABILITY,
FITNESS FOR A PARTICULAR PURPOSE AND NONINFRINGEMENT. IN NO EVENT SHALL THE
AUTHORS OR COPYRIGHT HOLDERS BE LIABLE FOR ANY CLAIM, DAMAGES OR OTHER
LIABILITY, WHETHER IN AN ACTION OF CONTRACT, TORT OR OTHERWISE, ARISING FROM,
OUT OF OR IN CONNECTION WITH THE SOFTWARE OR THE USE OR OTHER DEALINGS IN THE
SOFTWARE.

\section{Environment options}
The environment \texttt{timeline} has five optional arguments:
\begin{center}
\begin{lstlisting}
	\begin{timeline}
		[$\parameter{InnerBackgroundColor}$]
		[$\parameter{InnerForegroundColor}$]
		[$\parameter{OuterBackgroundColor}$]
		[$\parameter{OuterForegroundColor}$]
		[$\parameter{ConnectorColor}$]
\end{lstlisting}
\end{center}

They define the respective color for the events on the timeline.
Default settings are some orange \tikz{\node[fill=orange!75!white] at (0,0) {};} with black text for the inner part and some blueish \tikz{\node[fill=uniblue!75!white] at (0,0) {};} with white text for the outer part. 
If the fifth parameter for the connector color is not specified, then the $\parameter{InnerBackgroundColor}$ will be used instead.
Feel free to change the colours via the parameters.

\section{Commands}
Inside the \texttt{timeline} environment one can use the following commands.
\begin{description}
	\item[Add a timebar] This is done via the 6-parameter command
		%{startsymbol}{startX}{fromyear}{toyear}{intervallength}{endsymbol}
\begin{lstlisting}
\timebar{$\parameter{startsymbol}$}
	{$\parameter{startXcoordinate}$}
	{$\parameter{fromYear}$}
	{$\parameter{toYear}$}
	{$\parameter{stepLength}$}
	{$\parameter{endSymbol}$}
\end{lstlisting}
		$\parameter{startSymbol}$ can be any symbol one could put on the left side of `\verb'-'' at \verb'\draw[-]' in the \texttt{tikz}-command.
		
		$\parameter{startXcoordinate}$ is the $x$-coordinate. $y$ is always considered to be $0$.
		
		$\parameter{fromYear}$ is in \texttt{YYYY}.
		
		$\parameter{toYear}$ is in \texttt{YYYY}.
		
		$\parameter{stepLength}$ is the length of one year on the timebar.
		
		$\parameter{endSymbol}$ can be any symbol one could put on the right side of `\verb'-'' at \verb'\draw[-]' in the \texttt{tikz}-command.
	\item[Add a crunched bar part.] Used to literally skip some parts in the timebar where nothing happens. Drawn via the command
\begin{lstlisting}
	\zigzag{$\parameter{xStartCoordinate}$}
\end{lstlisting}
$\parameter{xStartCoordinate}$ is the $x$-coordinate where it should start. Usually, when using this command, one has to calculate the $x$-coordinate by hand from what was executed before.
Then length of the zigzag is $0.4$.

\item[Entries.] An entry has four possible commands, depending on whether it should appear above, below, shifted above, or shifted below. The unshifted versions are essentially the same.
\begin{lstlisting}
	\entry{$\parameter{year}$}{$\parameter{what}$}{$\parameter{who}$}
	\flipentry{$\parameter{year}$}{$\parameter{what}$}{$\parameter{who}$}
\end{lstlisting}
$\parameter{year}$ is \texttt{YYYY} or \texttt{YYYY-MM} which has to exists on the timebar. 
Such labels are automatically created in the process of executing a \texttt{timebar}-command. 
More precisely, you have to execute a \texttt{timebar}-command with $\parameter{fromYear}$ as $Y_1$ and $\parameter{toYear}$ as $Y_2$ and then can use an \texttt{entry}/\texttt{flipentry}/\texttt{entryshift}/\texttt{flipentryshift}-command with $\parameter{year}$ which is has to be between $Y_1$ and $Y_2$.
Otherwise it will create a missing label error.

$\parameter{what}$ is the content of the inner (near the timebar) box.

$\parameter{who}$ is the content of the outer box.

\begin{lstlisting}
	\entryshift{$\parameter{year}$}{$\parameter{what}$}{$\parameter{who}$}{$\parameter{distance}$}
	\flipentryshift{$\parameter{year}$}{$\parameter{what}$}{$\parameter{who}$}{$\parameter{distance}$}
\end{lstlisting}

$\parameter{distance}$ is the length (often specified in mm) which shifts the entry to the right (in direction of the $x$-coordinate).
\end{description}

\section{Example}
A small example is provided to demonstrate the commands:

\begin{lstlisting}
\begin{timeline}
  \timebar{|}{0}{1959}{1960}{.5}{}
  \zigzag{1}
  \entry{1959}{Branching Quantifiers}{Henkin}
  
  \timebar{}{1.4}{1989}{1990}{.5}{}
  \zigzag{2.4}
  \entry{1989}{IF-logic}{Hintikka, Sandu}
  
  \timebar{}{2.8}{1997}{2018}{.5}{stealth'}
  \entry{1997}{Compositional Semantics for IF}{Hodges}
  \entry{2003}{IF modal logic}{Tulenheimo}
  \entry{2007}{Dependence Logic}{Väänänen}
  \flipentry{2008}{Modal Dependence Logic}{Väänänen}
  \entry{2011-6}{Inclusion \& Exclusion Logic}{Galliani}
  \entry{2012-8}{Independence Logic}{Grädel, Väänänen}
  \flipentry{2011}{Modal Team Logic}{Müller}
  \entry{2016}{Multiteam Semantics}{Durand et al.}
  \flipentry{2016}{Prop. Team Logic}{Yang, Väänänen}
  \flipentryshift{2017}{Polyteam Semantics}{Hannula et al.}{2mm}
\end{timeline}
\end{lstlisting}
and produces the following timeline:

	\begin{timeline}
	 	\timebar{|}{0}{1959}{1960}{.5}{}
	  	\zigzag{1}
	  	\entry{1959}{Branching Quantifiers}{Henkin}
	  	
	  	\timebar{}{1.4}{1989}{1990}{.5}{}
	  	\zigzag{2.4}
		\entry{1989}{IF-logic}{Hintikka, Sandu}
	
		\timebar{}{2.8}{1997}{2018}{.5}{stealth'}
		\entry{1997}{Compositional Semantics for IF}{Hodges}
		\entry{2003}{IF modal logic}{Tulenheimo}
		\entry{2007}{Dependence Logic}{Väänänen}
		\flipentry{2008}{Modal Dependence Logic}{Väänänen}
        \entry{2011-6}{Inclusion \& Exclusion Logic}{Galliani}
	    \entry{2012-8}{Independence Logic}{Grädel, Väänänen}
		\flipentry{2011}{Modal Team Logic}{Müller}
		\entry{2016}{Multiteam Semantics}{Durand et al.}
		\flipentry{2016}{Propositional\\ Team Logic}{Väänänen\\Yang}
		\flipentryshift{2017}{Polyteam Semantics}{Hannula et al.}{3mm}
  	\end{timeline}

\section{Acknowledgements}
Thanks fly out to: Paul Gaborit.

\end{document}